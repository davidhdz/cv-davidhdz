%Copyright (C) 2012-2016 David Hernández <david.vzla@gmail.com>

%Permission is hereby granted, free of charge, to any person obtaining a copy
%of this software and associated documentation files (the "Software"), to deal
%in the Software without restriction, including without limitation the rights
%to use, copy, modify, merge, publish, distribute, sublicense, and/or sell
%copies of the Software, and to permit persons to whom the Software is
%furnished to do so, subject to the following conditions:

%The above copyright notice and this permission notice shall be included in
%all copies or substantial portions of the Software.

%THE SOFTWARE IS PROVIDED "AS IS", WITHOUT WARRANTY OF ANY KIND, EXPRESS OR
%IMPLIED, INCLUDING BUT NOT LIMITED TO THE WARRANTIES OF MERCHANTABILITY,
%FITNESS FOR A PARTICULAR PURPOSE AND NONINFRINGEMENT. IN NO EVENT SHALL THE
%AUTHORS OR COPYRIGHT HOLDERS BE LIABLE FOR ANY CLAIM, DAMAGES OR OTHER
%LIABILITY, WHETHER IN AN ACTION OF CONTRACT, TORT OR OTHERWISE, ARISING FROM,
%OUT OF OR IN CONNECTION WITH THE SOFTWARE OR THE USE OR OTHER DEALINGS IN
%THE SOFTWARE.

\documentclass[11pt,letterpaper,sans]{moderncv}   % opciones posibles incluyen tamaño de fuente ('10pt', '11pt' and '12pt'), tamaño de papel ('a4paper', 'letterpaper', 'a5paper', 'legalpaper', 'executivepaper' y 'landscape') y familia de fuentes ('sans' y 'roman')

%\usepackage[spanish]{babel}

% temas de moderncv
\moderncvtheme[orange]{casual}                % las opciones de estilo son 'casual' (por omision),'classic', 'oldstyle' y 'banking'
                                              % opciones de color 'blue' (por omision), 'orange', 'green', 'red', 'purple', 'grey' y 'black'
%\renewcommand{\familydefault}{\sfdefault}    % para seleccionar la fuente por omision, use '\sfdefault' para la fuente sans serif, '\rmdefault' para la fuente roman, o cualquier nombre de fuente
\nopagenumbers{}                              % elimine el comentario para suprimir la numeracion automatica de las paginas para CVs mayores a una pagina

% codificacion de caracteres
\usepackage[utf8]{inputenc}                   % reemplace con su codificacion
%\usepackage{CJKutf8}                         % si necesita usa CJK para redactar su CV en chino, japones o coreano

% ajustes para los margenes de pagina
\usepackage[scale=0.80]{geometry}
\setlength{\hintscolumnwidth}{2.7cm}           % si desea cambiar el ando de la columna para las fechas

% datos personales
\firstname{David A.}                          
\familyname{Hern\'andez Aponte}
\title{Curr\'iculum V\'itae}                  % dato opcional, elimine la linea si no desea el dato
%\address{Urb. Campo de Oro, Blq. 26, Apto.01-04, Sta. Juana}{M\'erida 5101 Venezuela}    % dato opcional, elimine la linea si no desea el dato
\mobile{+58 (416) 474 9784}                   % dato opcional, elimine la linea si no desea el dato
%\phone{+58 (274) 263 0640}                    % dato opcional, elimine la linea si no desea el dato
%\fax{+58(000)123 4567}                       % dato opcional, elimine la linea si no desea el dato
\email{david.vzla@gmail.com}                  % dato opcional, elimine la linea si no desea el dato
\homepage{davidhdz.github.io}    			  % dato opcional, elimine la linea si no desea el dato
%\extrainfo{informaci\'on adicional}          % dato opcional, elimine la linea si no desea el dato
%\photo[64pt][0pt]{13803334.jpg}               % '64pt' es la altura a la que la imagen debe ser ajustada, 0.4pt es grosor del marco que lo contiene (eliga 0pt para eliminar el marco) y 'picture' es el nombre del archivo; dato opcional, elimine la linea si no desea el dato
%\quote{Alguna cita (opcional}                % dato opcional, elimine la linea si no desea el dato

% para mostrar etiquetas numericas en la bibliografia (por omision no se muestran etiquetas), solo es util si desea incluir citas en en CV
%\makeatletter
%\renewcommand*{\bibliographyitemlabel}{\@biblabel{\arabic{enumiv}}}
%\makeatother

% bibliografia con varias fuentes
%\usepackage{multibib}
%\newcites{book,misc}{{Libros},{Otros}}
%----------------------------------------------------------------------------------
%            contenido
%----------------------------------------------------------------------------------

\begin{document}
%\begin{CJK*}{UTF8}{gbsn}                     % para redactar el CV en chino usando CJK
\maketitle

\section{Formaci\'on Acad\'emica}

%\cventry{a\~no--a\~no}{Grado}{Instituci\'on}{Ciudad}{\textit{Grade}}{Descripci\'on}  % Los argumentos del 3 al 6 pueden permanecer vacios
\cventry{2004--2018}{Ingeniero de Sistemas}{Universidad de Los Andes}{M\'erida, Venezuela}{}{}
\cventry{1996--1998}{Ingeniero Forestal (3\hspace{-1.5mm}$\phantom{a}^{\circ}$ semestre)}{Universidad de Los Andes}{M\'erida, Venezuela}{}{}
\cventry{1991--1996}{Técnico Medio Industrial mención Electrónica}{Escuela Técnica Industrial ``Manuel Antonio Pulido Méndez''}{M\'erida, Venezuela}{}{}

%\section{Tesis de maestr\'ia}
%\cvitem{t\'itulo}{\emph{T\'itulo}}
%\cvitem{sinodares}{Sinodales}
%\cvitem{descripci\'on}{Una breve descripci\'on de la tesis}

\section{Experiencia Laboral}
\cventry{{\footnotesize marzo--diciembre}\\2016--2017}{Director de Desarrollo}{Fundación Centro Nacional de Desarrollo e Investigación en Tecnologías Libres (CENDITEL)}{M\'erida, Venezuela}{}{Director de los proyectos:  Administración de redes, Cumaco TDA ,Canal de retorno para la Televisión Digital Abierta (TDA), Flujo de Transporte TDA, Caribay TDA (ARM), Tropicalia, Terepaima, Control Maestro TDA, Criptografía, Tibisay v2, RedSur, CorreoSur, Curso Metodología de Desarrollo, Autana, SIGESIC v2.0}
\cventry{{\footnotesize enero--actualidad}\\2016--2017}{Miembro del equipo editorial de la Revista Conocimiento y Licenciamiento Libre (CLIC)}{Fundación Centro Nacional de Desarrollo e Investigación en Tecnologías Libres (CENDITEL)}{M\'erida, Venezuela}{}{Editor, fotógrafo y maquetador en las revistas CLIC y Trampiando}
\cventry{{\footnotesize marzo--actualidad}\\2014--2017}{Analista de Desarrollo en Tecnologías Libres}{Fundación Centro Nacional de Desarrollo e Investigación en Tecnologías Libres (CENDITEL)}{M\'erida, Venezuela}{}{Desarrollador en los proyectos:  Infogobierno, CONVITE, Tropicalia, Canaima Caribay GNU/Linux, Bicentenario, Yojama, Caribay TDA, Aplicaciones interactivas para Televisión Digital Abierta (TDA).}
\cventry{{\footnotesize febrero--diciembre}\\2014}{Comunicador}{Colectivo Social Manovuelta}{M\'erida, Venezuela}{}{Comunicador popular en el programa Manovuelta Guerrilla Radio}
\cventry{{\footnotesize noviembre}\\2014}{Fotógrafo}{Abax Soluciones}{M\'erida, Venezuela}{}{Sesión fotográfica para la página web del Hotel Xu y Restaurant Enping Xu}
\cventry{{\footnotesize agosto}\\2014}{Contrato por honorarios profesionales}{Hilda Grassi}{M\'erida, Venezuela}{}{Desarrollo de módulos en R y SAGA-GIS para el cálculo aritmético de capas Raster a partir de imágenes en secuencia para el Proyecto 2012001862 ``Producto y proceso para medir el efecto antibiótico sobre organismos en un tiempo corto y utilizando biospeckle'' del Fondo Nacional de Ciencia, Tecnología e Innovación, FONACIT}
\cventry{{\footnotesize octubre--diciembre}\\2013}{Técnico de Sistemas y Operador de Audio}{Cuica 89.7 FM}{M\'erida, Venezuela}{}{Instalación de sistemas libres en la estación y operador de sonido para el programa La Clave del Barrio}
\cventry{{\footnotesize septiembre--octubre}\\2013}{Fotografía Fija y Media Manager}{Ecosur Films}{M\'erida, Venezuela}{}{Proyecto: Ciudad de Mérida, Galería a Cielo Abierto}
\cventry{{\footnotesize enero--mayo}\\2012}{Contrato por honorarios profesionales}{Fundación Centro Nacional de Desarrollo e Investigación en Tecnologías Libres (CENDITEL)}{M\'erida, Venezuela}{}{Desarrollo de la Distribución Canaima Caribay para Medios Comunitarios}
\cventry{{\footnotesize septiembre--octubre}\\1998--2011}{Beca trabajo}{Centro de Simulación y Modelos - Universidad de Los Andes}{M\'erida, Venezuela}{}{Mantenimiento, Soporte, Redes, Administración de servidores}
\cventry{{\footnotesize mayo--noviembre}\\2008--2009}{Consultor Tecnológico}{Grupo de Tecnologías Educativas - Corporación Parque Tecnológico de Mérida}{M\'erida, Venezuela}{}{Consultor para el uso de tecnologías libres - Portal Cero Dificultad}
\cventry{{\footnotesize agosto}\\2008}{Facilitador comunitario de Software Libre}{Coordinación Unidad de Software Libre y Formación de Recursos Humanos - Corporación Parque Tecnológico de Mérida}{M\'erida, Venezuela}{}{Instructor del curso ``Facilitador Comunitario de Software Libre''}
\cventry{{\footnotesize junio--agosto}\\1996}{Pasante}{Laboratorio de Controles y Procesos de la Escuela de Ingeniería de Sistemas - Universidad de Los Andes}{M\'erida, Venezuela}{}{Técnico en Electrónica - Programa de pasantías}

\section{Formaci\'on Complementaria}
\cventry{2016}{Intro to Python for Data Science}{DataCamp}{4h}{09/03/2016}{Certificate id: 1e14bab689c1cfbbabeed3b901edbea4797e1689}
\cventry{2015}{Introduction to R}{DataCamp}{4h}{10/15/2015}{Certificate id: 3b6278ade60c1f9988cc016534c61c2aacbeab00}
\cventry{2015}{Fotografía - Composición Fotográfica}{Educate Cursos online \& Udemy}{1.5h}{29/10/2015}{Certificado: UC-T4VVT8LV}
\cventry{2015}{Programación de Aplicaciones en Lenguaje Python}{Academia Software Libre - FUNDACITE - Mérida}{40h}{21/09/2015 -- 02/10/2015}{}
\cventry{2015}{\LaTeX}{Academia Software Libre - FUNDACITE - Mérida}{12h}{30/07/2015 - 03/07/2015}{}
\cventry{2013}{Taller de ``Educación en Comunicación Popular para la Producción de Contenidos''}{Ministerio del Poder Popular para la Comunicación e Información}{16h}{17/10/2013~-~18/10/2013}{}
\cventry{2005}{Curso de redes ``CCNA 1 Networking Basics''}{Cisco System Networking Academy / Universidad de Los Andes}{70h}{09/09/2005}{}
\cventry{2005}{Taller ``OPENOFFICE para soporte técnico''}{Universidad de Los Andes / Corporación Parque Tecnológico de Mérida}{32h}{27/04/2005}{}
\cventry{2002}{Curso de inglés ``English as a Second Language - The Óptimo Program''}{Tompkins Cortland Community College / VEN-USA Institute}{380h}{24/06/2002}{}

%\pagebreak{}

\section{Presentaci\'on en Conferencias y Congresos}
\cventry{2015}{Relator}{}{Plan Nacional de Tecnologías de Información 2016--2019}{Mérida, Venezuela}{\textit{Comisión Nacional de las Ciencias de Información}}
\cventry{2015}{Conferencista}{Un nuevo sabor de Canaima GNU/Linux: Caribay TDA}{Jornada Proyectos de Investigación y Desarrollo en el Área de Tecnologías Libres}{Mérida, Venezuela}{\textit{Fundación Centro Nacional de Desarrollo e Investigación en Tecnologías Libres (CENDITEL)}}
\cventry{2015}{Facilitador}{Taller introductorio a GNU/Linux}{}{Mérida, Venezuela}{Escuela Básica de la Facultad de Ingeniería de la Universidad de Los Andes}
\cventry{2014}{Participante}{Mesa de Canaima}{Jornada Canaima | 1er Encuentro Técnico Canaima Educativo}{Caracas, Venezuela}{\textit{Centro Nacional de Tecnologías de Información}}
\cventry{2014}{Cartel}{Proyecto Canaima Bicentenario}{3er Congreso Venezolano de Ciencia, Tecnología e Innovación}{Caracas, Venezuela}{\textit{Observatorio Nacional de Ciencia, Tecnología e Innovación}}
\cventry{2014}{Participante}{Mesa Única de Conceptualización}{8va Cayapa Canaima}{Bailadores, Venezuela}{\textit{Comunidad Canaima GNU/Linux}}
\cventry{2014}{Ponente en Expoferia}{Proyectos de la Fundación CENDITEL}{II Encuentro Regional de Ciencia, Tecnología e Innovación}{Coro, Venezuela}{\textit{Observatorio Nacional de Ciencia, Tecnología e Innovación}}
\cventry{2012}{Participante}{Mesa de Plataforma Tecnológica y Estructura de Canaima}{6ta Cayapa Canaima}{Barinas, Venezuela}{\textit{Comunidad Canaima GNU/Linux}}
\cventry{2012}{Ponente}{ULAnix Scientia regresa}{8vo Congreso Nacional de Software Libre}{Mérida, Venezuela}{\textit{Proyecto GNU de Venezuela}}
\cventry{2011}{Participante}{Mesa Única de Sabores}{Mini Cayapa Canaima Sabores}{Mérida, Venezuela}{\textit{Comunidad Canaima GNU/Linux}}
\cventry{2010}{Ponente}{Canaima Medios Comnitarios, Experiencias en la creación de una distribución Linux para radios comunitarias}{4ta Cayapa Canaima}{San Juan de los Morros, Venezuela}{\textit{Comunidad Canaima GNU/Linux}}
\cventry{2008}{Ponente}{ULAnix Scientia una distribución para el estudiante de Ingeniería y Ciencias}{FLISoL2008 (Festival Latinoamericano de Instalación de Software Libre)}{Mérida, Venezuela}{\textit{Comunidad de Software Libre Mérida}}
\cventry{2008}{Ponente}{ULAnix Scientia una distribución para el estudiante de Ingeniería y Ciencias}{VIII y IX Encuentro con la Física, Química, Matemática y Biología}{Mérida, Venezuela}{\textit{Facultad de Ciencias - Universidad de Los Andes}}
\cventry{2007}{Cartel}{ULAnix Scientia: Software a la medida de Científicos y Tecnólogos }{II Encuentro Nacional de Actores de Popularización de la Ciencia}{Mérida, Venezuela}{\textit{Ministerio del Poder Popular para Ciencia y Tecnología / Fundacite}}
\cventry{2007}{Ponente}{Escritorios en Linux: La magia de Beryl}{Ciclo de Charlas - Evento Cultura libre}{Ejido, Venezuela}{\textit{Grupo de Usuarios de Software Libre del Instituto Universitario Tecnológico de Ejido}}
\cventry{2007}{Ponente}{Escritorios en Linux: La magia de Beryl}{Contacto Universitario con el Software Libre}{Mérida, Venezuela}{\textit{Grupo de Usuarios de Software Libre de la Universidad de Los Andes}}
\pagebreak{}

\section{Organizaci\'on de Eventos}
%\cvlistitem{Colaborador y organizador de varios eventos relacionados con el Software Libre en la ciudad de Mérida, como: Festival Latinoamericano de Instalación de Software Libre (FLISoL), Python Day, Document Freedom Day, Software Freedom Day, Ubuntu Hour y más}
\cventry{Noviembre 2017}{Organizador}{X Jornadas Nacionales de Soberanía Tecnológica}{Colectivo Teletriunfador / Fundación CENDITEL}{}{Mérida, Venezuela}
\cventry{Noviembre 2017}{Organizador}{VII Jornadas de Apropiación Social de las Tecnologías Libres}{Fundación CENDITEL}{}{Mérida, Venezuela}
\cventry{Abril 2017}{Organizador}{FLISoL 2017 (Festival Latinoamericano de Instalación de Software Libre)}{Comunidad Software Libre Mérida}{}{Mérida, Venezuela}
\cventry{Noviembre 2016}{Organizador}{VI Jornadas Pertinencia Social de las Tecnologías Libres: Experiencias}{Fundación CENDITEL}{}{Mérida, Venezuela}
\cventry{Noviembre 2016}{Organizador}{Foro: Investigación y Desarrollo en Tecnologías Libres: Presente y Futuro en Venezuela y Nuestra América}{Fundación CENDITEL}{}{Mérida, Venezuela}
\cventry{Febrero 2016}{Organizador}{PyTatuy 2016 - Día Python Mérida}{Fundación Python Venezuela}{}{Mérida, Venezuela}
\cventry{Julio 2015}{Organizador}{I Jornada en Seguridad Informática. Perspectivas sociales, técnicas y jurídicas desde una visión venezolana}{Fundación CENDITEL}{}{Mérida, Venezuela}
\cventry{Abril 2012}{Organizador}{FLISoL 2012 (Festival Latinoamericano de Instalación de Software Libre)}{Comunidad Software Libre Mérida}{}{Mérida, Venezuela}
\cventry{Marzo 2012}{Colaborador}{DFD 2012 (Document Freedom Day)}{Comunidad LibreOffice Venezuela}{}{Mérida, Venezuela}
\cventry{Octubre 2011}{Colaborador}{Agile Tour Mérida 2011}{Agile Tour}{}{Mérida, Venezuela}
\cventry{Septiembre 2011}{Organizador}{SFD 2011 (Software Freedom Day - Día de la Libertad del Software)}{Comunidad Software Libre Mérida}{}{Mérida, Venezuela}
\cventry{Abril 2011}{Organizador}{FLISoL 2011 (Festival Latinoamericano de Instalación de Software Libre)}{Comunidad Software Libre Mérida}{}{Mérida, Venezuela}
\cventry{Abril 2010}{Organizador}{FLISoL 2010 (Festival Latinoamericano de Instalación de Software Libre)}{Comunidad Software Libre Mérida}{}{Mérida, Venezuela}
\cventry{Junio 2009}{Organizador}{3er Aniversario Ubuntu Venezuela}{Comunidad Ubuntu Venezuela}{}{Mérida, Venezuela}
\cventry{Abril 2009}{Organizador}{FLISoL 2009 (Festival Latinoamericano de Instalación de Software Libre)}{Comunidad Software Libre Mérida}{}{Mérida, Venezuela}
\cventry{Julio 2008}{Colaborador}{2do Aniversario Ubuntu Venezuela}{Comunidad Ubuntu Venezuela}{}{Caracas, Venezuela}
\cventry{Mayo 2008}{Colaborador}{4to Congreso Nacional de Software Libre}{Proyecto GNU de Venezuela}{}{Mérida, Venezuela}
\cventry{Abril 2008}{Organizador}{FLISoL 2008 (Festival Latinoamericano de Instalación de Software Libre)}{Comunidad Software Libre Mérida}{}{Mérida, Venezuela}
\cventry{Junio 2007}{Colaborador}{3er Congreso Nacional de Software Libre}{Proyecto GNU de Venezuela}{}{Mérida, Venezuela}
\cventry{Abril 2007}{Organizador}{FLISoL 2007 (Festival Latinoamericano de Instalación de Software Libre)}{Comunidad Software Libre Mérida}{}{Mérida, Venezuela}
\cventry{Octubre 2006}{Organizador}{I Jornadas de Software Libre}{Grupo de Usuarios de Software Libre de la Universidad de Los Andes}{}{Mérida, Venezuela}
\cventry{Julio 2005}{Organizador}{Festival de Instalación Linux}{Asociación de Estudiantes de Sistemas}{}{Mérida, Venezuela}

\section{Reconocimientos}
\cventry{2015}{IX Premio ``Cultor o Cultora, de la Tecnología Libre'' en Desarrollo e Investigación}{Fundación Centro Nacional de Desarrollo e Investigación en Tecnologías Libres}{Noviembre 2015}{}{}
\cventry{2015}{IX Premio a Equipo promotor de Red de la Tecnología Libre}{Fundación Centro Nacional de Desarrollo e Investigación en Tecnologías Libres}{Noviembre 2015}{}{}
\cventry{2014}{IV Premio ``Novicio del Año''}{Fundación Centro Nacional de Desarrollo e Investigación en Tecnologías Libres}{Noviembre 2014}{}{}


%\section{Investigaci\'on y Publicaciones}

%\cventry{2016}{Hernández, D. (2016)}{De Autómatas y Cimarrones: una reseña sobre ``La Ética del Hacker y el Espíritu de la Información'' de Pekka Himanen}{Revista CLIC Conocimiento Libre y Licenciamiento Nº 13. pp.203--209}{}{Recuperado: \url{https://convite.cenditel.gob.ve/revistaclic/index.php/revistaclic/article/download/856/809}}
%\cventry{}{Báez, N. Vera, J. Osorio, A. Guarisma, A. Hernández, D. (2016)}{RedSUR para una Economía Productiva en Venezuela}{Revista CLIC Conocimiento Libre y Licenciamiento Nº 13. pp.160--171}{}{Recuperado: \url{https://convite.cenditel.gob.ve/revistaclic/index.php/revistaclic/article/download/854/807}}
%\cventry{}{Colina, L. Ibarguen, Y. Hernández, L. Chourio, L. Ramírez, H. Hernández, D. (2016)}{Transmisión de Aplicaciones Interactivas para Televisión Digital Abierta en Venezuela}{Revista CLIC Conocimiento Libre y Licenciamiento Nº 13. pp.149--159}{}{Recuperado: \url{https://convite.cenditel.gob.ve/revistaclic/index.php/revistaclic/article/download/853/806}}
%\cventry{2007}{Dávila, J. Carrero, J. Molina, J. Díaz, G, Hernández, D.}{ULAnix Scientia: Software a la medida de Científicos y Tecnólogos}{Memorias, II Encuentro Nacional de Actores de Popularización de la Ciencia. pp.133--134}{}{Recuperado: \url{http://www.fundacite-merida.gob.ve/images/doc/documentos/memorias/memoria3.pdf}}
%\cventry{}{Hernández, D. (2007)}{Introducción a GNU Octave. Versión 1.0}{}{}{Recuperado: \url{https://github.com/davidhdz/ulanix-scientia-docs/blob/master/usr/share/ulanix-scientia-docs/octave/octave.pdf?raw=true}}

\section{Servicio comunitario}
\cventry{2016}{Capacitación sobre el manejo de herramientas se Software Libre en instituciones públicas del Estado Mérida}{Universidad de Los Andes}
{Capacitación del personal a cargo de la revista Conocimiento y Licenciamiento Libre (CLIC) de la Fundación Centro Nacional de Desarrollo e investigación en Tecnologías Libres (CENDITEL) en el uso de \LaTeX\ para la elaboración de plantillas y optimización del flujo de trabajo para la edición de revistas}{}{Repositorio: \url{https://github.com/davidhdz/Plantilla-CLIC}}

\section{Idiomas}
\cvlanguage{Español}{Nativo}{}
\cvlanguage{Inglés}{Nivel medio}{}

%\section{Capacidades}
%\cvlistitem{Trabajo en equipo}
%\cvlistitem{Trabajo bajo presión}
%\cvlistitem{Proactivo}
%\cvlistitem{Centrado en cumplir las metas}
%\cvlistitem{Enfocado en la resolución de problemas}

\pagebreak
\renewcommand{\cvcomputer}[2]{\cvline{#1}{\small#2}}
\section{Conocimientos}
\cvcomputer{Sistemas Operativos}{Gnu/Linux, Windows, MS-DOS, Mac OS X.}
\cvcomputer{Administrador}{Web, FTP, Correo.}
\cvcomputer{Ofimática}{Procesadores de texto, Hojas de cálculo, Presentaciones.}
\cvcomputer{Hardware}{Ensamble de equipos, diagnóstico de fallas, mantenimiento preventivo del computador.}
\cvcomputer{Programación}{Python, C/C++ (y librerías GTK,QT).}
\cvcomputer{Programación Científica}{Octave, R, Maxima.}
\cvcomputer{Maquetación}{\LaTeX, Scribus.}
\cvcomputer{Manipulación de imágenes}{Gimp, Inkscape.}
\cvcomputer{Diseño Web}{HTML, HTML5, XML, CSS, Wiki.}
\cvcomputer{Base de datos}{MySQL, PostgreSQL.}
\cvcomputer{Servidores}{Apache, Nginx.}
\cvcomputer{Fotografía}{Paisajismo, documental, callejera, periodismo, inmobiliaria (interiores/exteriores).}

%\section{Inter\'eses}
%\cvitem{Fotografía}{Entusiasta de la fotografía callejera, documental y paisajística}
%\cvitem{Senderísmo}{Senderísmo y acampada}

\section{Grupos y Asociaciones}
\cventry{2014--actualidad}{Miembro}{Colectivo Social Manovuelta}{}{}{\url{http://manovuelta.org.ve}}
\cventry{2014--actualidad}{Miembro}{Fundación Python Venezuela}{}{}{\url{http://www.python.info.ve/}}
%\cventry{2014--actualidad}{Miembro}{Frente Bolivariano de Innovadores, Investigadores y Trabajadores de la Ciencia}{}{}{\url{http://frebin.org.ve/}}
%\cventry{2013 - actualidad}{Miembro}{Colectivo F.U.G.A. (Fuerza de Guerrilla Audiovisual)}{}{}{}
\cventry{2007--actualidad}{Miembro}{Comunidad Ubuntu Venezuela}{}{}{\url{http://www.ubuntu.org.ve/}}
\cventry{2006--actualidad}{Miembro}{Grupo de Usuarios de Software Libre de la Universidad de Los Andes}{}{}{\url{http://www.coactivate.org/projects/gusla/project-home/}}
\cventry{2006--actualidad}{Colaborador}{ULAnix (Metadistribución GNU/Linux desarrollada por la Universidad de Los Andes)}{}{}{\url{http://nux.ula.ve}}



%\section{Extra 1}
%\cvlistitem{Tema 1}
%\cvlistitem{Tema 2}
%\cvlistitem{Tema 3}

%\renewcommand{\listitemsymbol}{-~}            % para cambiar el simbolo para las listas

%\section{Extra 2}
%\cvlistdoubleitem{Tema 1}{Tema 4}
%\cvlistdoubleitem{Tema 2}{Tema 5\cite{book1}}
%\cvlistdoubleitem{Tema 3}{}

% Las publicaciones tomadas de un archivo de BibTeX sin usar multibib
%  para etiquetas numericas: \renewcommand*{\bibliographyitemlabel}{\@biblabel{\arabic{enumiv}}}
%  para redefinir la cabecera ("Publicaciones"): \renewcommand{\refname}{Articles}

\renewcommand{\refname}{Publicaciones}
\nocite{*}
\bibliographystyle{unsrt}
\bibliography{pubs.bib}                   % 'publications' es el nombre del archivo BibTeX

% Las publicaciones tomadas de un archivo BibTeX usando el paquete multibib
%\section{Publicaciones}
%\nocitebook{book1,book2}
%\bibliographystylebook{plain}
%\bibliographybook{publications}              % 'publications' es el nombre del archivo BibTeX
%\nocitemisc{misc1,misc2,misc3}
%\bibliographystylemisc{plain}
%\bibliographymisc{publications}              % 'publications' es el nombre del archivo BibTeX

%\clearpage\end{CJK*}                          % si esta redactando su CV en chino usando CJK, \clearpage es requerido por fancyhdr para que funcione correctamente con CJK, aunque esto eliminara la numeracion de pagina al dejar \lastpage como no definido

\end{document}

%% end of file `cv-davidhdz.tex'.
