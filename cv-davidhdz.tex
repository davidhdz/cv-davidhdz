%Copyright (C) 2012 David Hernández <david.vzla@gmail.com>

%Permission is hereby granted, free of charge, to any person obtaining a copy
%of this software and associated documentation files (the "Software"), to deal
%in the Software without restriction, including without limitation the rights
%to use, copy, modify, merge, publish, distribute, sublicense, and/or sell
%copies of the Software, and to permit persons to whom the Software is
%furnished to do so, subject to the following conditions:

%The above copyright notice and this permission notice shall be included in
%all copies or substantial portions of the Software.

%THE SOFTWARE IS PROVIDED "AS IS", WITHOUT WARRANTY OF ANY KIND, EXPRESS OR
%IMPLIED, INCLUDING BUT NOT LIMITED TO THE WARRANTIES OF MERCHANTABILITY,
%FITNESS FOR A PARTICULAR PURPOSE AND NONINFRINGEMENT. IN NO EVENT SHALL THE
%AUTHORS OR COPYRIGHT HOLDERS BE LIABLE FOR ANY CLAIM, DAMAGES OR OTHER
%LIABILITY, WHETHER IN AN ACTION OF CONTRACT, TORT OR OTHERWISE, ARISING FROM,
%OUT OF OR IN CONNECTION WITH THE SOFTWARE OR THE USE OR OTHER DEALINGS IN
%THE SOFTWARE.

\documentclass[11pt,letterpaper,sans]{moderncv}   % opciones posibles incluyen tamaño de fuente ('10pt', '11pt' and '12pt'), tamaño de papel ('a4paper', 'letterpaper', 'a5paper', 'legalpaper', 'executivepaper' y 'landscape') y familia de fuentes ('sans' y 'roman')

% temas de moderncv
\moderncvtheme[orange]{casual}                % las opciones de estilo son 'casual' (por omision),'classic', 'oldstyle' y 'banking'
                                              % opciones de color 'blue' (por omision), 'orange', 'green', 'red', 'purple', 'grey' y 'black'
%\renewcommand{\familydefault}{\sfdefault}    % para seleccionar la fuente por omision, use '\sfdefault' para la fuente sans serif, '\rmdefault' para la fuente roman, o cualquier nombre de fuente
\nopagenumbers{}                              % elimine el comentario para suprimir la numeracion automatica de las paginas para CVs mayores a una pagina

% codificacion de caracteres
\usepackage[utf8]{inputenc}                   % reemplace con su codificacion
%\usepackage{CJKutf8}                         % si necesita usa CJK para redactar su CV en chino, japones o coreano

% ajustes para los margenes de pagina
\usepackage[scale=0.75]{geometry}
%\setlength{\hintscolumnwidth}{3cm}           % si desea cambiar el ando de la columna para las fechas

% datos personales
\firstname{David A.}                          
\familyname{Hern\'andez Aponte}
\title{Curr\'iculum V\'itae}                  % dato opcional, elimine la linea si no desea el dato
\address{Urb. Campo de Oro, Blq. 26, Apto.01-04, Sta. Juana}{M\'erida 5101 Venezuela}    % dato opcional, elimine la linea si no desea el dato
\mobile{+58 (416) 474 9784}                   % dato opcional, elimine la linea si no desea el dato
\phone{+58 (274) 263 0640}                    % dato opcional, elimine la linea si no desea el dato
%\fax{+58(000)123 4567}                       % dato opcional, elimine la linea si no desea el dato
\email{david.vzla@gmail.com}                  % dato opcional, elimine la linea si no desea el dato
%\homepage{www.davidhdz.com}                  % dato opcional, elimine la linea si no desea el dato
%\extrainfo{informaci\'on adicional}          % dato opcional, elimine la linea si no desea el dato
%\photo[64pt][0.4pt]{picture}                 % '64pt' es la altura a la que la imagen debe ser ajustada, 0.4pt es grosor del marco que lo contiene (eliga 0pt para eliminar el marco) y 'picture' es el nombre del archivo; dato opcional, elimine la linea si no desea el dato
%\quote{Alguna cita (opcional}                % dato opcional, elimine la linea si no desea el dato

% para mostrar etiquetas numericas en la bibliografia (por omision no se muestran etiquetas), solo es util si desea incluir citas en en CV
%\makeatletter
%\renewcommand*{\bibliographyitemlabel}{\@biblabel{\arabic{enumiv}}}
%\makeatother

% bibliografia con varias fuentes
%\usepackage{multibib}
%\newcites{book,misc}{{Libros},{Otros}}
%----------------------------------------------------------------------------------
%            contenido
%----------------------------------------------------------------------------------

\begin{document}
%\begin{CJK*}{UTF8}{gbsn}                     % para redactar el CV en chino usando CJK
\maketitle

\section{Formaci\'on Acad\'emica}

%\cventry{a\~no--a\~no}{Grado}{Instituci\'on}{Ciudad}{\textit{Grade}}{Descripci\'on}  % Los argumentos del 3 al 6 pueden permanecer vacios
\cventry{1998 - 2013}{Ingeniero de Sistemas (9\hspace{-1.5mm}$\phantom{a}^{\circ}$ semestre)}{Universidad de Los Andes}{M\'erida, Venezuela}{}{}
\cventry{1996 - 1998}{Ingeniero Forestal (3\hspace{-1.5mm}$\phantom{a}^{\circ}$ semestre)}{Universidad de Los Andes}{M\'erida, Venezuela}{}{}
\cventry{1991 - 1996}{Técnico Medio Industrial mención Electrónica}{Escuela Técnica Industrial "Manuel Antonio Pulido Méndez"}{M\'erida, Venezuela}{}{}

%\section{Tesis de maestr\'ia}
%\cvitem{t\'itulo}{\emph{T\'itulo}}
%\cvitem{sinodares}{Sinodales}
%\cvitem{descripci\'on}{Una breve descripci\'on de la tesis}

\section{Formaci\'on Complementaria}
\cventry{2005}{Curso de redes "CCNA 1 Networking Basics"}{Cisco System Networking Academy / Universidad de Los Andes}{70h}{09/09/2005}{}
\cventry{2005}{Taller "OPENOFFICE para soporte técnico"}{Universidad de Los Andes / Corporación Parque Tecnológico de Mérida}{32h}{27/04/2005}{}
\cventry{2002}{Curso de inglés "English as a Second Language - The Óptimo Program"}{Tompkins Cortland Community College / VEN-USA Institute}{380h}{24/06/2002}{}

\section{Experiencia Laboral}
\cventry{2011-2012}{Desarrollador}{Fundación Centro Nacional de Desarrollo e Investigación en Tecnologías Libres}{M\'erida, Venezuela}{}{Desarrollo de la Distribución Canaima Caribay para Medios Comunitarios}
\cventry{1998-2011}{Técnico}{Centro de Simulación y Modelos - Universidad de Los Andes}{M\'erida, Venezuela}{}{Mantenimiento, Soporte, Redes, Adminsitración de servidores}
\cventry{2008-2009}{Consultor Tecnológico}{Grupo de Tecnologías Educativas - Corporación Parque Tecnológico de Mérida}{M\'erida, Venezuela}{}{Consultor para el uso de tecnologías libres - Portal Cero Dificultad}
\cventry{2008}{Instructor}{Coordinación Unidad de Software Libre y Formación de Recursos Humanos - Corporación Parque Tecnológico de Mérida}{M\'erida, Venezuela}{}{Instructor del curso "Facilitador Comunitario de Software Libre"}
\cventry{1996}{Pasante}{Laboratorio de Controles y Procesos de la Escuela de Ingeniería de Sistemas - Universidad de Los Andes}{M\'erida, Venezuela}{}{Técnico en Electrónica - Programa de pasantías}

\pagebreak{}

%\section{Investigaci\'on y Publicaciones}
%\cventry{2013}{Proyecto de Grado}{TITULO}{Universidad de Los Andes}{Abril 2013}{}

\section{Presentaci\'on en Conferencias y Congresos}
\cventry{2012}{Participante}{6ta Cayapa Canaima}{Barinas, Venezuela}{\textit{Comunidad Canaima GNU/Linux}}{Mesa de Plataforma Tecnológica y Estructura de Canaima}
\cventry{2012}{Ponente}{8vo Congreso Nacional de Software Libre}{Mérida, Venezuela}{\textit{Proyecto GNU de Venezuela}}{ULAnix Scientia regresa}
\cventry{2011}{Participante}{Mini Cayapa Canaima Sabores}{Mérida, Venezuela}{\textit{Comunidad Canaima GNU/Linux}}{}
\cventry{2010}{Ponente}{4ta Cayapa Canaima}{San Juan de los Morros, Venezuela}{\textit{Comunidad Canaima GNU/Linux}}{Canaima Medios Comnitarios, Experiencias en la creación de una distribución Linux para radios comunitarias}
\cventry{2008}{Ponente}{FLISoL2008 (Festival Latinoamericano de Instalación de Software Libre)}{Mérida, Venezuela}{\textit{Comunidad de Software Libre Mérida}}{ULAnix Scientia una distribución para el estudiante de Ingeniería y Ciencias}
\cventry{2008}{Ponente}{VIII y IX Encuentro con la Física, Química, Matemática y Biología}{Mérida, Venezuela}{\textit{Facultad de Ciencias - Universidad de Los Andes}}{ULAnix Scientia una distribución para el estudiante de Ingeniería y Ciencias}
\cventry{2007}{Cartel}{II Encuentro Nacional de Actores de Popularización de la Ciencia}{Mérida, Venezuela}{\textit{Ministerio del Poder Popular para Ciencia y Tecnología / Fundacite}}{ULAnix Scientia: Software a la medida de Científicos y Tecnólogos }
\cventry{2007}{Ponente}{Ciclo de Charlas - Evento Cultura libre}{Ejido, Venezuela}{\textit{Grupo de Usuarios de Software Libre del Instituto Universitario Tecnológico de Ejido}}{Escritorios en Linux: La magia de Beryl}
\cventry{2007}{Ponente}{Contacto Universitario con el Software Libre}{Mérida, Venezuela}{\textit{Grupo de Usuarios de Software Libre de la Universidad de Los Andes}}{Escritorios en Linux: La magia de Beryl}

\section{Organizaci\'on de Eventos}
\cventry{2012}{Organizador}{FLISoL2012(Festival Latinoamericano de Software Libre)}{Comunidad Software Libre Mérida}{}{Mérida, Venezuela}
\cventry{2012}{Colaborador}{DFD2012(Document Freedom Day)}{Comunidad LibreOffice Venezuela}{}{Mérida, Venezuela}
\cventry{2011}{Colaborador}{Agile Tour Mérida 2011}{Agile Tour}{}{Mérida, Venezuela}
\cventry{2011}{Organizador}{SFD2011(Software Freedom Day - Día de la Libertad del Software)}{Comunidad Software Libre Mérida}{}{Mérida, Venezuela}
\cventry{2011}{Organizador}{FLISoL2011(Festival Latinoamericano de Software Libre)}{Comunidad Software Libre Mérida}{}{Mérida, Venezuela}
\cventry{2010}{Organizador}{FLISoL2010(Festival Latinoamericano de Software Libre)}{Comunidad Software Libre Mérida}{}{Mérida, Venezuela}
\cventry{2009}{Organizador}{3er Aniversario Ubuntu Venezuela}{Comnidad Ubuntu Venezuela}{}{Mérida, Venezuela}
\cventry{2009}{Organizador}{FLISoL2009(Festival Latinoamericano de Software Libre)}{Comunidad Software Libre Mérida}{}{Mérida, Venezuela}
\cventry{2008}{Colaborador}{2do Aniversario Ubuntu Venezuela}{Comnidad Ubuntu Venezuela}{}{Caracas, Venezuela}
\cventry{2008}{Colaborador}{4to Congreso Nacional de Software Libre}{Proyecto GNU de Venezuela}{}{Mérida, Venezuela}
\cventry{2008}{Organizador}{FLISoL2008(Festival Latinoamericano de Software Libre)}{Comunidad Software Libre Mérida}{}{Mérida, Venezuela}
\cventry{2007}{Colaborador}{3er Congreso Nacional de Software Libre}{Proyecto GNU de Venezuela}{}{Mérida, Venezuela}
\cventry{2007}{Organizador}{FLISoL2007(Festival Latinoamericano de Software Libre)}{Comunidad Software Libre Mérida}{}{Mérida, Venezuela}
\cventry{2006}{Organizador}{I Jornadas de Software Libre}{Grupo de Usuarios de Software Libre de la Universidad de Los Andes}{}{Mérida, Venezuela}
\cventry{2005}{Organizador}{Festival de Instalación Linux}{Asociación de Estudiantes de Sistemas}{}{Mérida, Venezuela}

\pagebreak{}

\section{Idiomas}
\cvlanguage{Español}{Nativo}{}
\cvlanguage{Inglés}{Nivel medio}{}

\renewcommand{\cvcomputer}[2]{\cvline{#1}{\small#2}}
\section{Conocimientos}
\cvcomputer{Sistemas Operativos}{Gnu/Linux (Debian, Ubuntu, Red Hat/Fedora, Canaima), Windows(98, XP y Vista), MS-DOS}
\cvcomputer{Administrador}{Web, FTP, Correo.}
\cvcomputer{Ofimática}{LibreOffice/OpenOffice, MSOffice}
\cvcomputer{Hardware}{Ensamble de equipos, diagnóstico de fallas, mantenimiento preventivo del computador.}
\cvcomputer{Programación estructurada y OO}{Python, C/C++ (y librerías GTK,QT).}
\cvcomputer{Programación Científica}{Octave, R, Maxima.}
\cvcomputer{Maquetación}{\LaTeX, Scribus.}
\cvcomputer{Manipulación digital de imágenes}{Gimp, Inkscape}
\cvcomputer{Diseño Web}{HTML, XML, CSS, Wiki.}
\cvcomputer{Base de datos}{MySQL, PostgreSQL.}

%\section{Grupos y Asociaciones}
%\cvcomputer{2007 - actualidad}{Miembro de la Comunidad Ubuntu Venezuela}{}{}{}
%\cvcomputer{2006 - actualidad}{Miembro del Grupo de Usuarios de Software Libre de la Universidad de Los Andes}{}{}{}
%\cvcomputer{2006 - actualidad}{Colaborador en el Proyecto ULAnux (Metadistribución desarrollada por la Universidad de Los Andes)}{}{}{}

\section{Grupos y Asociaciones}
\cventry{2007 - actualidad}{Miembro}{Comunidad Ubuntu Venezuela}{}{}{}
\cventry{2006 - actualidad}{Miembro}{Grupo de Usuarios de Software Libre de la Universidad de Los Andes}{}{}{}
\cventry{2006 - actualidad}{Colaborador} {Proyecto ULAnux (Metadistribución desarrollada por la Universidad de Los Andes)}{}{}{}

\section{Inter\'eses}
\cvitem{Fotografía}{Entusiasta de la fotografía callejera, documental y paisajística}

%\section{Extra 1}
%\cvlistitem{Tema 1}
%\cvlistitem{Tema 2}
%\cvlistitem{Tema 3}

%\renewcommand{\listitemsymbol}{-~}            % para cambiar el simbolo para las listas

%\section{Extra 2}
%\cvlistdoubleitem{Tema 1}{Tema 4}
%\cvlistdoubleitem{Tema 2}{Tema 5\cite{book1}}
%\cvlistdoubleitem{Tema 3}{}

% Las publicaciones tomadas de un archivo de BibTeX sin usar multibib
%  para etiquetas numericas: \renewcommand*{\bibliographyitemlabel}{\@biblabel{\arabic{enumiv}}}
%  para redefinir la cabecera ("Publicaciones"): \renewcommand{\refname}{Articles}

%\nocite{*}
%\bibliographystyle{plain}
%\bibliography{publications}                   % 'publications' es el nombre del archivo BibTeX

% Las publicaciones tomadas de un archivo BibTeX usando el paquete multibib
%\section{Publicaciones}
%\nocitebook{book1,book2}
%\bibliographystylebook{plain}
%\bibliographybook{publications}              % 'publications' es el nombre del archivo BibTeX
%\nocitemisc{misc1,misc2,misc3}
%\bibliographystylemisc{plain}
%\bibliographymisc{publications}              % 'publications' es el nombre del archivo BibTeX

%\clearpage\end{CJK*}                          % si esta redactando su CV en chino usando CJK, \clearpage es requerido por fancyhdr para que funcione correctamente con CJK, aunque esto eliminara la numeracion de pagina al dejar \lastpage como no definido

\end{document}

%% end of file `davidhcv.tex'.
